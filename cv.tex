\documentclass[american]{cv-class}
\usepackage{afterpage}
\usepackage{hyperref}
\usepackage{color}
\usepackage{xcolor}
\usepackage[raggedrightboxes]{ragged2e}
\usepackage{siunitx}
\usepackage{datenumber, xfp, fp}
\usepackage[spanish]{babel}

\newcounter{dateone}%
\newcounter{datetwo}%
\setmydatenumber{dateone}{1994}{08}{07}%
\setmydatenumber{datetwo}{\the\year}{\the\month}{\the\day}%
\FPsub\result{\thedatetwo}{\thedateone}
\FPdiv\myage{\result}{365.2425}

\hypersetup{
    colorlinks=true,
    linkcolor=blue
}

%% sample.bib contains your publications
\addbibresource{sample.bib}

\RequirePackage{xcolor}
\definecolor{blue}{HTML}{6CE0F1}

\begin{document}
\header{Juan David}{ García}
{Analyst Developer Frontend-Backend}

\vspace{1.15cm}
\fcolorbox{white}{gray}{\parbox{\dimexpr\textwidth-2\fboxsep-2\fboxrule}{%
	.....
}}



\begin{aside}


    \vspace{3.2cm}
    
    \section{Sobre de mí}
    \justifying
    \begin{small}
    	{ Soy estudiante de noveno semestre de \textbf{ingeniería de sistemas} de la
     Universidad del Cauca.  Tengo conocimientos en \textbf{OOP} en \textbf{Java, C++ y Python }y \textbf{UML} con \textbf{PowerDesigner y StarUml}. He utilizado tecnologías  front-end como \textbf{Vue.js}; tecnologías de backend como \textbf{Django}, herramientas de bases de datos SQL como \textbf{SQLite, MySql, SQLServer y Oracle}. . Soy organizado, \textbf{autodidacta}, me gusta enfrentarme a nuevos retos que asumo con compromiso y responsabilidad.
    	}
    \end{small}

    
	\section{Contacto}
	\\
	\href{https://msng.link/o/?dawish7=tg}{\raisebox{-0.35ex}
	{\includegraphics[scale=0.0023]{img/telegram-logo.png}} (+57) 311 781 8794}
    
    \href{https://www.linkedin.com/in/juan-david-garcía-velasco-6674131a7}{\raisebox{-0.35ex}{\includegraphics[scale=0.006]{img/certifications/linkedin-logo.png}} David García}
    
	\href{mailto:jdgarcia216@unicauca.edu.co}{\raisebox{-0.35ex}{\includegraphics[scale=0.050]{img/institutions/unicauca-logo.png}} jdgarcia216}
    
    \href{https://github.com/jdavid777}{\raisebox{-0.35ex}{\includegraphics[scale=0.010]{img/github-logo.png}} jdavid777}
    
    \includegraphics[scale=0.030]{img/ubication-logo.png} Popayán, Colombia 
    
    
    \section{Idiomas}
    \\
    \begin{small}
    \textbf{Español Nativo}
    \\
    \textbf{Ingles: A2 }

    \end{small}
	
\section{Habilidades Blandas}

	\justifying
   \begin{small}
   \textbf{Autodidacta,}
    	{ \textbf{Trabajo en equipo}, Liderazgo, \textbf{Comunicación,} Creatividad, Empatía, Responsabilidad, Pensamiento Crítico, Resolución de Problemas.
    	}
    \end{small}
    
\section{Habilidades Técnicas}

	\justifying
    \begin{enumerate}
    
    \item \textbf{Java, Python, C++ }
    \item  \textbf{Javascript, CSS3, HTML5}
    \item  \textbf{Django, Vue.js, Graphql}
    \item  \textbf{MySQL, SQLite, Oracle}
    \item \textbf{Git,Github, Docker}
    
    \end{enumerate}
\end{aside}

 

%Cuerpo del documento


\section{Experiencia}
\begin{entrylist}

	\entry
	{Feb. 1 - Abr, 2020}
	{Analista de Requerimientos - Desarrollador} 
	{ \raisebox{-1.5ex}{\includegraphics[scale=0.05]{img/institutions/unicauca-logo.png} Universidad del Cauca, CO}}
	{\justifying 
	    Encargado de la \textbf{recolección de requisitos}, documentación y validación de las necesidades de los clientes, además del  \textbf{diseño de la arquitectura} de la aplicación web, responsable del desarrollo del módulo de autenticación (front-end y back-end) y el despliegue de la aplicación UXTest en Heroku.
	
	}
	\entry
	{Feb.20-Oct,15}
	{Desarrollador frontend}
	{ \raisebox{-1.5ex}{\includegraphics[scale=0.05]{img/institutions/unicauca-logo.png} Universidad del Cauca, CO}}
	{\justifying 
	Apoyo en la refinación y documentación de requisitos para el proyecto DeliveryFood, responsable del módulo de geolocalización en el cual se consumen diferentes librerias del \textbf{API de Google Maps} para la renderización de mapas, rutas, tiempos estimados, ubicaciones de pedidos, clientes y establecimientos, además, apoyé el desarrollo del modulo para la gestión de mensajeros, todo desde un cliente desarrollado en \textbf{Vue.js}. Se realizaron peticiones al back-end de la aplicación mediante un endpoint \textbf{Graphql.}}
	
	
\end{entrylist}

\section{Formación}
\begin{entrylist}
	\entry
	{2016 - Presente}
	{Ingeniero de Sistemas }
	{ \raisebox{-1.5ex}{\includegraphics[scale=0.05]{img/institutions/unicauca-logo.png} Universidad del Cauca, CO}}
	{\justifying Estudiante finalizando octavo semestre.
	}
	\\
	\entry
	{209 - 2014}
	{Bachiller académico }
	{ \raisebox{-1.5ex} {\includegraphics[scale=0.05]{img/institutions/colegioLogo.jpeg} Bachillerato Patía, CO}}
	{\justifying 
	}
	\end{entrylist}
	\section{Proyectos}
    \begin{entrylist}
    \entry
    	{Abr. 6, 2021}
    	{Algoritmo para planificación de redes}
    	{ \raisebox{-1.5ex}{\includegraphics[scale=0.05]{img/institutions/unicauca-logo.png} Universidad del Cauca, CO}}
    	{\justifying Desarrollo de algoritmo basado en el método \textbf{VLSM} 
    	para la planificación de redes usando el lenguaje de programacón \textbf{Python}.}
    \\
    \entry
    	{Presente}
    	{Trabajo de Grado}
    	{ \raisebox{-1.5ex}{\includegraphics[scale=0.05]{img/institutions/unicauca-logo.png} Universidad del Cauca, CO}}
    	{\justifying Iniciando una práctica profesional con la Universidad del Cauca para el programa "Estilos de Vida Saludable", el cual manejará programas deportivos (rutinas, planes de alimentación y entrenamiento) de las personas que se inscriban a los diferentes programas deportivos ofertados por la universidad. La práctica abarca desde la recolección de requisitos con los clientes (funcionarios del área deportiva de la universidad) hasta el despliegue del sistema. Para el proyecto se usan \textbf{Angular para el front-end y Springboot para el back-end; Oracle para la base de datos.}}
    	
    \end{entrylist}

\newpage
\section{Pasatiempos}
    \begin{itemize}
        \item
        {Tocar la guitarra, música, vídeojuegos}
        \item
        {Baloncesto, calistenia, natación}
        \\
    \end{itemize}



%\section{Certificaciones Ingenieria de Software}
%    \begin{entrylist}
%    	\entry
%    	{2020}
%    	{CCNA Routing and Switching: Introducción a redes }
%    	{ \raisebox{-1.5ex}{\includegraphics[scale=0.02]{img/certifications/Cisco-Logo-2006.png}}}{\href{url}{\raisebox{-0.35ex}{\includegraphics[scale=0.025]{img/certifications/certificate-logo.png}}{\addfontfeature{Color=blue}{\textit{View certificate}}}}}
    	
%    	\entry
%    	{2020}
%    	{CCNA Routing and Switching: Principios básicos de routing y switching }
%    	{ \raisebox{-1.5ex}{\includegraphics[scale=0.02]{img/certifications/Cisco-Logo-2006.png}}}{\href{url}{\raisebox{-0.35ex}{\includegraphics[scale=0.025]{img/certifications/certificate-logo.png}}{\addfontfeature{Color=blue}{\textit{View certificate}}}}}
    	
%    	\entry
%    	{2021}
%    	{Scrum Fundamentals Certified Credential}
%    	{ \raisebox{-1.5ex}{\includegraphics[scale=0.04]{img/certifications/scrumstudy-logo.jpg}}}{\href{url}{\raisebox{-0.35ex}{\includegraphics[scale=0.025]{img/certifications/certificate-logo.png}}{\addfontfeature{Color=blue}{\textit{View certificate}}}}}
    %\end{entrylist}

\section{Referencias}
\begin{entrylist}
	
	\entry
	{Academic}
	{Msc.Eng. Jorge Adrian Muñoz}
	{System Engineer} 
	{Phone: +573216072893, Popayán, CO.}

	\entry
	{Academic}
	{PhD. MSc. Eng. Hendrys Fabián Tobar}
	{System Engineer} 
	{Phone: +573184477258, Popayán, CO.}
	
	\entry
    {Personal}
	{Cristian Collazos}
	{Web Developer} 
	{Phone: +573187670415, Popayán, CO.}


\end{entrylist}

\vspace{0.5cm}
\begin{center}
	\emph{For all legal purposes I certify that all
the answers and information written down by me,
in this resume are true (C.S.T.
	Art.62 num. 1st)}
\end{center}

\begin{flushright}
	%\emph{{\includegraphics[scale=0.4]{img/firma.JPG}}}
\end{flushright}
\begin{flushright}
	\emph{\textbf{Juan David García Velasco}}
\end{flushright}
\begin{flushright}
	\emph{C.C. 1059915147. El Bordo-Patía, CO}
\end{flushright}

\end{document}
